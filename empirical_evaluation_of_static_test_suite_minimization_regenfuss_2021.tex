\documentclass[a4paper,10pt]{article}

\usepackage[utf8]{inputenc}
\usepackage[english]{babel}
\usepackage{algorithmicx}
\usepackage{algpseudocode}
\usepackage{amsmath}
\usepackage{dutchcal}
\usepackage{graphicx}
\usepackage{hyperref}
\usepackage{natbib}

\title{Empirical Evaluation of Test Suite Reduction}
\author{Adrian Regenfuß}

\begin{document}

\maketitle

%TODO: create table of contents

\begin{abstract}
As a response to ever-growing test suites with long runtimes, several
different approaches have been developed to reduce the time for test
suite execution to give useful results. One of these approaches is test
suite reduction: selecting a sample of tests that maximizes coverage on
the tested source code. This paper attempts to replicate the findings
in \citealt{cruciani2019scalable}, which borrows techniques from big
data to handle very large test suites. We use independently generated
testing data from open source projects, and find that TODO.
\end{abstract}

\section{Introduction}

\section{Related Work}

\section{Handling Large Test Suites}

\subsection{Methods}

\subsubsection{Test Case Selection}

\subsubsection{Test Case Priorization}

\subsubsection{Test Suite Reduction}

\subsection{Algorithms for Reduction}

\subsubsection{Greedy Selection}

%Needs coverage information

\subsubsection{Clustering}

%Very indirect.

\subsubsection{Searching}

\subsubsection{Hybrids}

\subsection{Classifying the FAST algorithms}

\section{Preparation}

\subsection{Gathering Data}

\subsubsection{Selecting Projects}

%Medium-Sized Java Projects

\subsubsection{Generating Coverage Information}

%Only Line Coverage
%Using Teamscale Jacoco Agent

\subsubsection{Collecting Fault Coverage Information}

%Using Pitest

\subsubsection{Combining Tests Suites}

\section{Method}

\section{Results}

\subsection{Test Suite Reduction}

\subsection{Fault Detection Loss}

\subsection{Running Time}

\subsection{Comparison to ''Scalable Approaches to Test Suite Reduction''}

\section{Limitations}

\section{Future Work}

\section{Summary}

\bibliographystyle{plainnat}
\bibliography{sources}

\end{document}
