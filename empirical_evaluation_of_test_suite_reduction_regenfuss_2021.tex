\documentclass[a4paper,10pt]{article}

\usepackage[utf8]{inputenc}
\usepackage[english]{babel}
\usepackage{algorithmicx}
\usepackage{algpseudocode}
\usepackage{amsmath}
\usepackage{dutchcal}
\usepackage{graphicx}
\usepackage{hyperref}
\usepackage{natbib}

\title{Empirical Evaluation of Test Suite Reduction}
\author{Adrian Regenfuß}

\begin{document}

\maketitle

\tableofcontents
\newpage

\begin{abstract}
As a response to ever-growing test suites with long runtimes, several
different approaches have been developed to reduce the time for test
suite execution to give useful results. One of these approaches is test
suite reduction: selecting a sample of tests that maximizes coverage on
the tested source code. This paper attempts to replicate the findings
in \citealt{cruciani2019scalable}, which borrows techniques from big
data to handle very large test suites. We use independently generated
testing data from open source projects, and find that TODO.
\end{abstract}

\section{Introduction}

3 pages

To prevent the introduction (or re-introduction) of bugs, software
developers often test their software after making changes to it.
This is known as regression testing, and takes up a significant portion
of development cost. However, the resulting test suites can grow quite
significantly in size and execution time, which hinders development
speed and increases costs.

This work attempts to determine how different test suite reduction
strategies compare to each other in terms of time performance, fault
detection loss and magnitude of reduction.

\section{Terms and Definitions}

\section{Related Work}

5 pages

\subsection{Handling Large Test Suites}

\subsubsection{Test Case Selection}

\subsubsection{Test Case Priorization}

\subsubsection{Test Suite Reduction}

\subsection{Algorithms for Reduction}

\subsubsection{Greedy Selection}

Needs coverage information

\subsubsection{Clustering}

\subsubsection{Searching}

\subsubsection{Hybrids}

\section{Approach}

8 pages

\subsection{Replicating ''Scalable Approaches for Test Suite Reduction''}

1 page

\subsection{Algorithms Used}

7 pages

\subsubsection{FAST}

4 pages

\paragraph{FAST++}

\paragraph{FAST-all}

\paragraph{FAST-CS}

\paragraph{FAST-pw}

\subsubsection{Adaptive Random Testing}

2 pages

\paragraph{ART-D}

\paragraph{ART-F}

\subsubsection{Greedy Algorithm}

$\frac{1}{2}$ a page

\subsubsection{Random Selection}

$\frac{1}{2}$ a page

New method, after \citealt[p. 17]{khan2018systematic}

\section{Evaluation}

17 pages

\subsection{Research Questions}

\paragraph{Research Question 1: Does the Relative Effectiveness of the Different Algorithms Replicate the Results in \citealt{cruciani2019scalable}?}

\subparagraph{Research Question 1.1: Does Their Relative Effectiveness in TSR Replicate the Findings in \citealt{cruciani2019scalable}?}

\subparagraph{Research Question 1.2: Does Their Relative Effectiveness in FDL Replicate the Findings in \citealt{cruciani2019scalable}?}

\paragraph{Research Question 2: Does the Relative Runtime Performance of the Different Algorithms Replicate the Results in \citealt{cruciani2019scalable}?}

\paragraph{Research Question 3: How Much Better Than Random Selection are Specialized Algorithms?}

\paragraph{(Possibly) Research Question 4: Do the Results in \citealt{cruciani2019scalable} Replicate with the Original Test Data?}

\subsection{Study Setup}

\subsection{Study Objects}

\subsubsection{Selecting Projects}

Medium-Sized Java Projects

assertj-core, commons-lang, commons-math, commons-collections, jopt-simple, jsoup

\subsubsection{Generating Coverage Information}

Only Line Coverage

Using Teamscale Jacoco Agent

\subsubsection{Collecting Fault Coverage Information}

Using Pitest

\subsubsection{Combining Tests Suites}

\subsection{Results}

\paragraph{Research Question 1.1}

\paragraph{Research Question 1.2}

\paragraph{Research Question 2}

\paragraph{Research Question 3}

\paragraph{(Possibly) Research Question 4}

\subsubsection{Running Time}

\subsection{Discussion}

\subsubsection{Comparison to ''Scalable Approaches to Test Suite Reduction''}

\subsection{Threats to Validity}

\subsubsection{Conclusion Validity}

\subsubsection{Internal Validity}

\subsubsection{Construct Validity}

\subsubsection{External Validity}

\section{Future Work}

1 page

\section{Summary}

2 pages

\newpage

\bibliographystyle{plainnat}
\bibliography{sources}

\end{document}
